% 
% Annual Cognitive Science Conference
% Sample LaTeX Paper -- Proceedings Format
% 

% Original : Ashwin Ram (ashwin@cc.gatech.edu)       04/01/1994
% Modified : Johanna Moore (jmoore@cs.pitt.edu)      03/17/1995
% Modified : David Noelle (noelle@ucsd.edu)          03/15/1996
% Modified : Pat Langley (langley@cs.stanford.edu)   01/26/1997
% Latex2e corrections by Ramin Charles Nakisa        01/28/1997 
% Modified : Tina Eliassi-Rad (eliassi@cs.wisc.edu)  01/31/1998
% Modified : Trisha Yannuzzi (trisha@ircs.upenn.edu) 12/28/1999 (in process)
% Modified : Mary Ellen Foster (M.E.Foster@ed.ac.uk) 12/11/2000
% Modified : Ken Forbus                              01/23/2004
% Modified : Eli M. Silk (esilk@pitt.edu)            05/24/2005
% Modified: Niels Taatgen (taatgen@cmu.edu)  10/24/2006

%% Change ``a4paper'' in the following line to ``letterpaper'' if you are
%% producing a letter-format document.






%%%% Remaining todos
% finish up introduction
% write discussion

% overall proofreading


\documentclass[10pt,letterpaper]{article}

\usepackage{cogsci}
\usepackage{pslatex}
\usepackage{apacite}
\usepackage{graphicx}
\usepackage{color}
\usepackage{amsmath}

\definecolor{Red}{RGB}{255,0,0}
\newcommand{\red}[1]{\textcolor{Red}{#1}}  

\title{ When does a near-miss sting? }
 
\author{{\large \bf Desmond C. Ong (dco@stanford.edu)} \\
{\large \bf Jamil Zaki (jzaki@stanford.edu)} \\
{\large \bf Noah D. Goodman (ngoodman@stanford.edu)} \\
  Department of Psychology, Stanford University, Stanford CA, USA 
}

\begin{document}

\maketitle

\begin{abstract}
Observers often judge agents to be more unhappy when the agents miss a desired outcome by a small margin, compared to a large margin. This has been shown in situations where the agents have control over their situation, such as in missing planes. Here, we show that intuitive theories of emotion incorporate near-misses even for random events, and this might be due in part to the illusion of control over random events. Finally, we integrate near-misses into a more general model of affective cognition, and quantify the psychological cost of a near-miss relative to winning and losing.


%Near misses---missing a desired outcome by a small margin---are more emotionally intense than normal misses, even though the outcomes tend to be no different, and people readily accord near-miss effects when reasoning about others. Yet there are still open questions: what are the distance dimensions along which near misses are judged, and how do people incorporate near misses into more general affective cognition, or reasoning about emotion? In this paper we show that intuitive theories of emotion seem to weigh near-misses even for random events, and this is driven by the semblance of action-outcome contingency. Finally, we incorporate near misses into a more general model of affective cognition, and quantify the psychological cost of a near-miss relative to winning and losing.

\textbf{Keywords:} 
Near Miss; Counterfactual Distance; Lay Theories; Emotion
\end{abstract}


\begin{quote}
\textit{``Close only counts in horseshoes and hand grenades"} 
--- English Idiom
\end{quote}

Evidence suggests that contrary to the above idiom, close counts---\textit{emotionally}.


	When Rob achieves a desired outcome, such as winning a soccer match, we intuitively know that he would feel positive. Conversely, when Rob loses, or otherwise fails to achieve an outcome, we can reason that he would feel negative. If he just fails to achieve the outcome by a small margin, a \textit{near-miss}, such as losing a soccer game by a single goal, we intuitively recognize that he actually would feel worse than if he had missed by a larger margin, because the outcome was ``so close" to winning. Penalty shootouts in soccer provides the most exaggerated of such near-miss scenarios: the losing team often loses because of one missed ball, sometimes an inch shy of escaping the goalkeeper's hands. These details add much more emotional intensity to all agents involved, more so than just a simple loss. Contrary to the above idiom, close counts---\textit{emotionally}.

% maybe introduce affective cognition
%
%	Such intuitive reasoning about emotions comprise what we call \textit{affective cognition}, or reasoning about emotions and other affective states (some citations xxx, \cite{OngAffCog}). ... 
	
	Psychologists have long known that near-misses form an integral, but surprisingly not well understood part of affective cognition \cite{Johnson1986, Gleicher1990}. Most of this work falls under the broader category of counterfactual thinking, or thinking about ``what might have been" \cite{Bryne2002,McMullen2002, Medvec1997, Roese1997}. Near-miss counterfactuals are particularly engaging, because these possible worlds almost happened: they were separated from the current world by some small ``distance". Consider \citeA{Kahneman1982}'s classic example of Mr Tees, who missed his plane by 5 minutes, and Mr Crane, who missed his plane by 30 minutes. Both men were delayed due to traffic (i.e., it was somewhat out of their control), but people consistently and reliably judge the person who narrowly missed his plane to feel much worse than the one who missed it by a wider margin. One proposed \textit{causally-relevant} explanation for the near miss effect is that of controllability: Mr Tees could easily think of actions he could have done differently (``if only I woke up ten minutes earlier") that would have led to him catching the plane. \red{NDG: controllability isn't explained clearly here, and no alternative is given, so it isn't clear why you mention it.}


%% Can this paragraph fit here? or should it belong between the end of the Intro and the start of Expt 1, in a section called "Predictions" (that was the way I had it initially).
%	Based upon findings from previous work \cite{Kahneman1982, Johnson1986, Gleicher1990}, we can list several predicted properties of near-miss effects. See Fig. \ref{PredictionFig} for an illustration, where we have included both near-miss and just-hit effects, although we focus on the former in this paper. First, the near-miss effect should be \textit{non-linear} with respect to distance to the desired outcome---the near miss effect should only occur at small distances, and should matter increasingly more with increasingly smaller distances. Second, the magnitude of the near-miss effect should be \textit{small but proportional} to the difference between the desired and undesired outcome.
%\red{NSG: these properties are predicted from your graph, but it's not clear where your graph came from -- it sounds like you're just making them up...}
%
%\begin{figure}[htb!]
%\includegraphics[width=\columnwidth]{images/predictionFig.png}
%\caption{ Prediction. A plot of emotional valence against ``distance" to the counterfactual world, where the unshaded region represents the desired outcome, and the shaded region, the ``miss" region. Near misses and ``just-hits" are predicted to be non-linear variations of valence at small distances to the miss-hit boundary.}
%\label{PredictionFig}
%\end{figure}



% be a little less vague 
%The distances that separate the desired-but-unattained counterfactual world from the current world.

\red{NDG: here starts the more precise stuff. note: the notion of emotion attribution and tie ins to ToE should have already appeared. need to be more precise about what you mean by causally-relevant.}
	There however, remains many open questions regarding the nature of the near-miss effect, and in this paper we address three of them. First, what are the dimensions of distance that observers judge to be relevant to an agent's emotions?  The answer most commonly proposed by the current literature asserts that people should consider causally-relevant dimensions, like the amount of time one misses a plane by. This would predict that people should not exhibit near miss effects in their lay theory when considering games of chance, or random events, as the causal process that generated the outcomes are based on chance. However, previous work has shown that gamblers persist more after near misses, showing a near-miss effect on motivation even though the outcome of games like slots are independent of the gambler's actions \cite{Kassinove2001, Reid1986}. In Experiment 1, we show that observers readily judge an agent who ``narrowly misses" on a die game (by rolling a number close to the target number) to feel worse than one who misses by a larger amount, even though outcomes on a die game are not ordered like the number line. This suggests that observers may also consider causally-irrelevant dimensions

%Observers seem to consider causally-irrelevant dimensions of distance as well, when a causally-relevant dimension does not exist.

\red{NDG: be clearer here: first, even if people are sensitive to causally-irrelevant dimensions, do they weight causally-relevant ones more? second, is this weighting flexible in the sense that it readily adjusts to contextual factors (not just e.g. physical factors)?}
Second, how do observers reason using these distance dimensions? In any particular situation, there could be more than one dimension of closeness, although not all of them would be causally-relevant, or otherwise relevant to the task at hand. Depending on the context, different dimensions of distance should matter to different extents. For example, if one was flipping over numbered cards trying to match a target number, is the proximal distance of the chosen card to the target card relevant? Is the numerical distance of the chosen card to the target card relevant? For the former, yes, and for the latter, much less so. What about if one was guessing a number and writing it down instead? In that case, numerical closeness should matter more. In Experiment 2, we show that observers are sensitive to contextual information (the rules of the game and the information that is presented) that changes the relevance of different distances in a random card-guessing task, and spontaneously alter their judgments during the task when presented with additional information.
\textcolor{red}{(to discuss with Noah more on this point. action-outcome contingency.)} %not sure how to add this in.

Finally, how much does a near-miss cost psychologically? That is, what is the size of the near miss effect (narrowly missing a desired outcome) relative to the utility \red{[NDG: not utility, happiness... check throughout.]} of actually obtaining said desired outcome? In Experiment 3, we build upon a previous model of affective cognition \cite{OngAffCog} using a gambling paradigm that allows us to parametrically vary features of the situation that affects judgment of emotions, such as the payoff structure and the distance to the neighboring outcomes. We explicitly incorporate modeling of near-miss effects into an existing quantitative model. This allows us to estimate relative effect sizes and gives us a better quantitative understanding of near miss judgments. More importantly, this allows the integration of near miss emotional judgments into existing models of affective cognition, and allows the construction of more comprehensive models of affective cognition.  



%Spell out relevant and irrelevant. Deeper theory.

%Nearness compared to actual outcome differences



%Hmm maybe not the best example: even if the 7 mins was reduced to 0, they wouldn't necessarily have won.

%	Argentina nearly won the 2014 FIFA World Cup Final, conceding the only goal of the match with barely 7 minutes left in extra time, but as the above idiom morbidly points out\footnote{Points in horseshoes are scored based on distance thrown horseshoes land from the target stake. Thrown hand grenades, in contrast, do not need to hit their target to be effective.}, close in this case, does not count. However close they were, they did not win. However, as Argentinian supporters would attest, close does matter---\textit{emotionally}.



%	Though we live in only one of many possible realizations of the world, our mental lives---and consequently, our emotional lives---are constantly spent exploring other possible worlds via counterfactual thinking \cite{Bryne2002, Gleicher1990, Johnson1986, Roese1997}. ``Near-miss" or close counterfactual comparisons in particular, are so mentally engaging because these possible worlds had almost happened. Consider \citeA{Kahneman1982}'s classic example of missing a plane by 5 minutes, as opposed to 30 minutes: people consistently and reliably judge the person who narrowly missed his plane to feel much worse than the one who missed it by a wider margin. One proposed reason is that it is much easier to generate possible counterfactual antecedents that would have resulted in the counterfactual consequent of catching the plane. The near-miss character could easily generate counterfactuals like ``If only I woke up 5 minutes earlier" or ``if only I had packed my bag the night before", that would result in the consequent ``then I would have caught my plane". If the counterfactual world is somehow \textit{closer} to the current world, then perhaps the counterfactual world would only require a smaller change in the causal chain that led up to the current world in order to be realized.
%
%	Previous research has identified some of the impact of closeness on counterfactual thought \cite{Kahneman1990, Teigen1996}. Closeness increases the activation of counterfactual thought, by increasing the salience of the counterfactual world \cite{Kahneman1982, Roese1997}, and additionally also amplifies the affective consequence of the counterfactual comparison \cite{Johnson1986, Kahneman1982}. Narrowly missing a plane or a World Cup Title feels far worse than missing it by a large margin. 
%	
%	Yet, there remains many open questions regarding the nature of these distances. What are the relevant dimensions of closeness that people incorporate into their lay theories of the world, and into their lay theories of emotion? Intuitively, people should consider only causally-relevant dimensions, like the amount of time one misses a plane by. However, anecdotally we are reminded of many instances where a (randomly-generated) lucky draw number is announced, and the holder of a (randomly-assigned) lucky draw ticket that is off by 1 number expresses extreme negative emotions. Given the random nature of this game, his number is not any ``closer" to reality than any other number in the set of possibilities.
	
	

%\subsection{Outline of paper:}
%\begin{enumerate}
%\item Lay out near miss predictions. Noticably: clearly, on both win and lose sides.
%\item Expt 1: just show it with vignettes, where distance is causally related to outcome
%\item Expt 2: show it with die vignettes, where distance is irrelevant
%\item Expt 3: card task, show that the relevant dimension can be tweaked
%\end{enumerate}
%
%
%
%\subsection{Contributions:}
%\begin{enumerate}
%\item ToE takes into account near misses -- but along what dimensions?
%\item show robustly that it considers both neg and positive misses (expt1)
%\item show that it considers irrelevant distances
%\item 
%\end{enumerate}
%
%Any similarity, causal counterfactual
%control-relevant (exploitable) causal counterfactual
%covert utility differences







%% Dice Vignette experiment





\section{Experiment 1: Near Miss effects in a random event}

In Experiment 1, we tested if participants would incorporate near-miss effects in their judgment of emotions when agents were playing a luck-based game, where the near-misses are along a causally-irrelevant dimension. Here we used a dice game; because the process of rolling dice is entirely random, the numerical closeness of one's outcome compared to the desired outcome is not relevant. \red{NDG: i tweaked this \P, but should be adjusted based on how we clarify causal relevance in intro.}

\subsubsection{Participants.} We recruited 150 participants through Amazon's Mechanical Turk.

\subsubsection{Procedures.} Participants read about two characters, Jacob and Alex, who were playing a gambling game, and both needed to roll a 6 on a die to win. Jacob rolled a 1, while Alex rolled a 5. Participants then answered attention check questions (``what did X roll?") before attributing emotions along six categories (\textit{happiness}, \textit{sadness}, \textit{relief}, \textit{regret}, \textit{contentment} and \textit{disappointment})  to each character. Finally, they answered a three-alternative forced choice question: ``Which character felt worse?", and were allowed to endorse ``they both felt equally bad" as an option. They were allowed to give a free-response justification for their choice.

\subsubsection{Results.} Three participants were excluded for failing the attention check. Participants rated the near miss character (the character who rolled the 5) as feeling significantly more disappointed ($t(146)=2.17, p=0.03$), but no different on the other emotions. In the forced-choice question, a large majority (107/147 = 73\%) rated both characters as feeling equally bad. Among the remaining participants, significantly more participants rated the character who rolled the 5 (the near miss character) as feeling worse (N=30) compared to the character who rolled a 1 (N=10; exact binomial test $p=.002$; bootstrapped simulation with 10,000 iterations on full sample, $p=0.0007$) (See Fig. \ref{Expt1ResultFig}.)

\begin{figure}[htb!]
\includegraphics[width=\columnwidth]{images/Expt1results.png}
\caption{ Expt 1 Results. Proportions of forced choice response. Error bars indicate standard errors. }
\label{Expt1ResultFig}
\end{figure}

To gain further insight into participants' judgments, we analyzed their free-response justifications and coded them into three categories. 84 (57\%) participants made their judgments based on equal outcomes (e.g. ``they both lost so they should feel equally bad"), 40 (27\%) participants made reference to closeness (e.g. ``he was soooo close"), and only 22 (15\%) participants made an explicit reference to there being no closeness difference (``it's a 1/6 chance for both of them"; ``the numbers are meaningless"). 1 participant chose not to give a justification. 


Thus, we find that while a large majority of participants said that both characters felt equally bad, this is primarily due to the fact that both characters lost. This is in line with our predictions \red{[NDG: i'll like this better when the predictions feel less ad hoc in intro..]} that the near-miss effect is much smaller in magnitude than the actual utility derived from the loss, and perhaps for these participants, the near miss difference was not above their threshold to endorse a difference in emotional valence. Interestingly, of the participants who made any remarks on closeness or the lack thereof, the majority actually remarked that there is a subjective feeling of closeness. This suggests that some observers are sensitive to near-miss effects in this scenario and the majority (irrationally) judge closeness based on a causally-irrelevant dimension.







\section{Experiment 2: Changing the relevant dimensions}

We designed Experiment 2 to dissociate closeness effects along different dimensions. Using a card guessing task, we manipulated the task-relevance of both the positions of the cards or the numbers written on the cards, and showed that near-miss effects along the task relevant-dimension are more strongly incorporated into observers' lay theory of emotion. %\red{NDG: contrary to expt 1??}

\subsubsection{Participants.} We recruited 200 participants through Amazon's Mechanical Turk, and assigned them to one of two conditions: Position-relevant (\textit{Pos}; N=100) and Number-relevant (\textit{Num}; N=100)
%, and Two-Step-Position (\textit{Two-Pos}; N=100).


\subsubsection{Procedures.} 

\begin{figure}[htb!]
\includegraphics[width=\columnwidth]{images/card_paradigm.png}
\caption{ Expt 2 Paradigm, \textit{Pos} condition. Characters' goal is to pick the target card, 10, outlined in purple. In the critical trial, the characters pick 19, in red, which is \textbf{close} in \textbf{distance}, and 11, in green, which is \textbf{close} in \textbf{number}. In other trials, one of the cards picked might be the number 1 (indicated by the blue arrow), which is \textbf{Far} in both distance and number. }
\label{Expt2ParadigmFig}
\end{figure}

\red{NDG: it's hard to follow this section because the condition labels, etc are cryptic. how about ``close distance'', ``far both'', ``close number''. and ``position relevant'', ``number relevant''? or something that ties more closely to control -- here it's controllability i think, not simple relevance?}

In the Position-relevant (\textit{Pos}) condition, participants saw a 5x4 array of cards face down. They were told that two characters were playing a game: the cards were numbered 1-20, and they had to pick the number 10 to win. There were three possible characters (of which participants only saw two): Scott, who picked 19 (close distance), Frank, who picked 11 (close number), and David, who picked 1 (far both). For example, in the trial depicted in Fig. \ref{Expt2ParadigmFig}, participants saw the close distance character and the close number character. After the characters picked their cards, the winning number 10 is revealed. Participants then rated the emotions of the two characters they saw (along the same six emotions as Expt 1). Finally, participants answered a forced-choice question, ``Who felt worse?", with the option to say ``Both felt equally bad." In total there are 3 possible pairings (``Close Distance vs. Close Number", ``Close Distance vs. Far Both", and ``Close Number vs. Far Both"), which are all between subject manipulations. Each participant only saw one trial.



In the Number-relevant (\textit{Num}) condition, we changed the rules of the game. There were 19 blank cards, and a target card (circled in purple). Characters were assigned a blank card and had to guess what the number was behind the target card, writing their answers on their assigned blank card. Thus, in the \textit{Pos} condition, the number of the goal was known (10) while the position was unknown, characters picked a position and were assigned a number (based on their choice); in this \textit{Num} condition, the position of the goal was known, but the number was unknown, and characters were assigned a position and picked a number. Importantly, the visual description that participants saw is similar to the Position-relevant condition. Thus, after characters wrote their guesses, the winning number behind the purple card is revealed (to be 10). Participants then attributed emotions to the two characters, and made a forced-choice judgment about who felt worse.

%\red{NDG: the Two-Pos condition is hard to follow.}
%The Two-Step-Position (\textit{Two-Pos}) condition was similar to the \textit{Pos} condition, except that after characters picked their cards but \textit{before} the winning card is revealed, participants make one set of emotion attributions and one forced-choice on who felt worse. Following this, the winning number 10 is revealed, and then participants make another set of attributions. Hence, participants in this condition made two sets of attributions, one before the location of the winning card is revealed (\textit{Two-Pos-BeforeReveal}), and once after (\textit{Two-Pos-AfterReveal}).


\subsubsection{Predictions.}
We predicted that in the \textit{Pos} condition, proximity would be judged to be a more relevant dimension of closeness than numerosity, and so the close distance character would be judged to feel worse than the close number character, although the close number character would, to a lesser extent, be judged to feel worse than the character who chose the far card. In the \textit{Num} condition, on the other hand, proximity is irrelevant, and so we predicted that the close number character would be judged to feel the worst, and there would be no difference between the close distance character and the far character. \red{NDG: why?? this doesn't seem to follow from any theory laid out so far in the paper.}

%The \textit{Two-Pos} condition has an interesting twist. Prior to finding out the position of the winning card, position is still a more relevant dimension than numerosity because of the context, but participants do not yet know the position of the winning card, which makes it impossible to judge closeness based on proximal distance. In this attribution, observers should make judgements based on numerosity. There would also be no difference between the proximally close and the far characters, and we predicted that the numerically close character will be judged to feel the worse of them all (i.e., \textit{Two-Pos-BeforeReveal} results should be similar to \textit{Num}). However, after finding out the position of the winning card, proximal closeness becomes possible to judge, and so we should expect to see the proximally close character being judged as feeling the worst (\textit{Two-Pos-AfterReveal} results should be similar to \textit{Pos}).



\begin{figure}[htb!]
\includegraphics[width=\columnwidth]{images/cardCombined_forcedWorse.png}
\caption{ Expt 2 Results. Proportions of forced choice response. Error bars indicate standard errors. Top row: \textit{Pos}ition-relevant condition. Bottom row: \textit{Num}ber-relevant condition.}
\label{Expt2ResultFig}
\end{figure}

\subsubsection{Results.} Thirteen participants were excluded for failing the attention checks.
First, let us consider the raw emotion attributions. In the \textit{Num} condition, as compared to the far character, the numerically close character was judged to feel more disappointed ($t(43)=2.81, p=.007$), more regret ($t(43)=2.44, p=.02$), more sadness ($t(43)=2.54, p=.015$) and less relief ($t(43)=2.12, p=.04$). For the \textit{Two-Pos-AfterReveal} attributions, the proximally-close character was judged to feel more disappointment ($t(31)=3.25, p=.003$), more regret ($t(31)=3.56, p=.001$), more sadness ($t(31)=2.76, p=.01$) and less happiness ($t(31)=2.67, p=.01$) compared to the numerically-close character. The proximally-close character was also judged to feel more disappointment ($t(33)=2.73, p=.01$), more regret ($t(33)=4.24, p=.0001$), more sadness ($t(33)=2.99, p=.005$), less happiness ($t(33)=2.49, p=.018$), and less relief ($t(33)=2.77, p=.009$) than the far character. All of these effects are in the predicted direction, but none of the other comparisons came out significant, suggesting that the near-miss effect might not be strong enough to be seen when comparing individual ratings of emotion.




The results for the forced-choice ratings are shown in Fig. \ref{Expt2ResultFig}. In line with our predictions, in the \textit{Pos} condition, the proximally-close character was judged to feel worse than the numerically-close character (bootstrapped simulation with 10,000 iterations on full sample, $p=.023$) and the far character (bootstrap $p=.0027$), and to a much smaller extent, the numerically-close character was judged to feel worse than the far character ($p=.016$). For the \textit{Two-Pos-AfterReveal} judgments, we find them to be qualitatively similar, in line with our predictions: the proximally-close character was judged to feel worse than the numerically-close character ($p=.0001$) and the far character (bootstrap $p=0$ as there were no observations for the far character feeling worse). The numerically-close character was not judged, however, to feel worse than the far character ($p=.41$).

By contrast, we see the opposite pattern of results in the \textit{Num} and \textit{Two-Pos-BeforeReveal} attributions. In the \textit{Num} condition, the numerically close character is judged to feel worse than the proximally-close character (bootstrap $p=.0017$) and the far character ($p<.0001$). The is no difference between the proximally-close and far characters ($p=.17$). In the \textit{Two-Pos-BeforeReveal} attributions, the numerically-close character is judged to feel worse than the proximally-close character ($p=.004$) and the far character ($p=.0001$), while there is no difference between the proximally-close and far characters ($p=.13$).


The results suggest that observers are sensitive to multiple dimensions, in this case, both proximal closeness and numerical closeness, and are able to flexibly judge which is the dimension that is more relevant to the task. When characters have to pick the physical location of the card, participants attribute near-miss effects along proximal closeness, and when characters have to choose a number (rather than a location), participants attribute near-miss effects along numerical closeness. However, participants still judge near misses along numerical closeness (but to a much smaller extent) when numerosity is not a task-relevant dimension, supporting the results from Experiment 1. 
%In the Two-Step-Before attribution, for example, numerical closeness is irrelevant, but the position of the winning card is unknown, so perhaps participants are judging near-misses based on the only information available.

\red{NDG: hmm. these results don't seem to add much to the previous lit, do they?}

\section{Experiment 3}
	Experiment 3 involved a meta-analysis of three prior experiments that were designed to examine the features underlying affective cognition in a gambling paradigm. 
	
\subsubsection{Participants and procedures.}
	690 participants were recruited across 3 different experiments previously reported in \citeA{OngAffCog}. The basic trial involves watching a character spin a wheel and win the amount on the wheel (Fig. \ref{Expt3ParadigmFig}). Participants then attributed 8 emotions (\textit{happy}, \textit{sad}, \textit{anger}, \textit{surprise}, \textit{disgust}, \textit{fear}, \textit{content} and \textit{disappointment}) to the character after the outcome on the wheel, using 9 point Likert scales. Each participant saw 10 trials, and the payoff and probability structure of the wheels were varied systematically to decorrelate the amount won with the expected value of the wheel. The first experiment only had these basic wheel trials: the second and third had these basic wheel trials intermixed with emotion attribution trials given other stimuli (faces and utterances) instead of wheels. We extracted data from the subset of wheel trials from the second and third experiment, and the entire first experiment, to amass a dataset of 3048 observations from 690 participants to conduct a meta-analysis on.
	
	These experiments were initially designed to test how different features of the situation, namely the amount won and the prediction error, affected participants' attribution of emotion to the character. Yet, because we randomized the position on which the spinner lands, these experiments incidentally provided a valuable dataset to test for a near-miss effect.

\begin{figure}[htb!]
\includegraphics[width=\columnwidth]{images/expt3Paradigm.png}
%\includegraphics[width=\columnwidth]{images/Expt3Distribution.png}
\caption{ Paradigm for the meta analysis reported in Expt 3. Participants attribute emotions to an agent after the outcome of a spin. After this spin, the agent won \$60 (as indicated by the black pointer), but almost won a lower amount. }
\label{Expt3ParadigmFig}
\end{figure}


\subsubsection{Previous model.} The model we built in \citeA{OngAffCog} incorporated three important features: the amount won, the prediction error (PE), and the absolute value of the prediction error. That is, the emotion attributed to the agent after event X was:
\begin{align}
E(X) &= b_0 + b_1 \text{win}(X) + b_2 \text{PE}(X) + b_3 \text{absPE}(X) + \epsilon \label{PEModel}
\end{align}
which was a linear combination of \textit{win}, \textit{PE} and \textit{absPE}. The absolute value of the PE was added to test for nonlinear effects, namely, loss aversion, whereby agents will be more sensitive to negative PE values than to positive PE values. More discussion can be found in the paper, but this was the starting point for the following model.


\subsubsection{Adding Near Miss to the model.}

%We started with the normalized ending position of where the spinner landed, which ranged from 0 to 1, with 0 indicating the boundary with the sector it would have landed in if the wheel spun less, and 1, the boundary with the sector it would have landed in if the wheel spun more (the wheels spun clockwise). 

Next, we proceeded to define a near miss distance. We calculated a normalized ``distance from the edge" which ranged from 0 to 0.5, with 0 being the boundary edge and 0.5 indicating the exact center of the sector. We then took a reciprocal transform ($1/x$) to introduce a non-linearity that favors smaller distances, and finally multiplied the transformed distance with the difference in payment amounts from the current sector to the next nearer sector. This last component was to account for the difference in utility in the two payouts. Hence, we had:
\begin{align}
NM(X) &= \frac{1}{\text{distanceToEdge}(X)} * \Delta\text{Payoffs}(X) \label{NMRegressor}
\end{align}
which we added to the model in Eqn. \ref{PEModel}. To illustrate, for the result shown in Fig. \ref{Expt3ParadigmFig}, the distance is approximately .05 (about 5\% of the sector size away from the \$25-\$60 boundary), and the $\Delta$Payoff is 60-25 = 35, as 25 is the next nearest sector.

\subsubsection{Meta analysis results.}

We fit a linear mixed-effects model with the amount won, PE, absPE, and the Near Miss (NM) term (Eqn. \ref{PEModel}, \ref{NMRegressor}) as fixed effects, and random intercepts by participant, wheel, and experiment. There is a significant slope on the NM term ($b = \text{-}3.5 * 10^{\text{-}5}, t(682)=\text{-}2.80, p=0.005$) on happiness. To understand the effect size of this term, let us consider the slope on win ($b = 0.0405, t(682) = 7.08, p<.0001$), PE ($b=0.036, t(682)=5.86, p<.0001$), and absPE ($b=\text{-}0.015, t(682) = \text{-}2.83, p=.005$) and the example in Figure \ref{Expt3ParadigmFig}. Not considering the near miss effect, and all else being equal, if the result had changed from \$60 to \$100, there will be an \textit{increase} in happiness due to $win$, $PE$ and $absPE$ of $40*(.0405+0.036+(\text{-}0.015)) = 2.46$ points on a 9 point Likert scale. In contrast, if we moved from the center of the \$60 sector to a distance of .01 (1\% of the sector size) away from the \$60/\$100 boundary, there would be a \textit{decrease} in happiness of $40*(1/0.5 - 1/0.01)*(\text{-}3.5 * 10^{\text{-}5}) = 0.137$ points on a 9 point scale. Thus, in this gambling scenario, the effect of a near miss on subjective happiness attributed is on the order of 5\% of the relative utility of winning the next higher amount. Getting a near miss on the \$60 wheel in Fig. \ref{Expt3ParadigmFig} and narrowly missing the sector \$100 (narrowly missing winning \$40 more) has a subjective cost equivalent to losing about \$2. This is a small effect relative to actually winning, yet it is  a large and not insignificant effect considering that this effect does not depend on changing actual payoffs, but relative closeness. Consider too, that this is a stylized game of chance, with fictional characters and hypothetical gambles, which all might lead to underestimating the size of the true effect in real-life situations like missing planes.

%\red{NDG: i like this result. clear, understandable, and contextualized. it may be that we'll want to include it in our psych rev revisions...}

\section{Discussion}

Near-misses matter emotionally, and in this paper we sought to understand how people factor near-misses into lay theories of emotion. First, we showed that in addition to considering causally-relevant dimensions, people seem to incorporate distances along causally-irrelevant dimensions, such as the number on a die roll---such numbers are generated by chance and numerical closeness has no actual relevance to the outcome, yet people seem to infer an illusory relevance. Second, when presented with multiple dimensions of closeness, people appropriately select task-relevant dimensions on which to base near-miss judgments. Third, we found evidence for a near miss effect across a meta-analysis of three experiments with a gambling paradigm designed to study features of affective cognition. This analysis allowed a comparison of the size of the near miss effect relative to the utility of actually winning the next alternative. Thus, in summary, we investigated certain properties of near-miss judgments (and of course, much more work needs to be done) and incorporated those findings into a larger, more complete model of affective cognition.


As we mentioned in the introduction, though near-miss effects and the broader class of counterfactual effects on emotion seem to be so intuitive, especially to the scientists studying them, there still remains many open questions. The results presented here suggests some interesting properties about near-miss judgments, and more work needs to be done to unify this and previous, more general work on counterfactual judgments into quantitative models of affective cognition. The model that we expanded in Experiment 3 already has some consideration of counterfactual judgement with the prediction error and loss aversion terms, yet it is still not the complete story. For example, related work in counterfactuals more generally has shown that the controllability of the counterfactual event \cite{Roese1995}, the temporal recency of the counterfactual event \cite{Miller1990}, and whether the outcome resulted from an act of omission or an act of commission \cite{Kahneman1982, Landman1987} all affect lay judgements of emotion. How might these features of counterfactuals interact with near-misses and fit into a computational model of affective cognition?







\section{Acknowledgments}

This work was supported in part by an A*STAR National Science Scholarship to DCO and by a James S. McDonnell Foundation Scholar Award to NDG.


\bibliographystyle{apacite}

\setlength{\bibleftmargin}{.125in}
\setlength{\bibindent}{-\bibleftmargin}

\bibliography{cfDistance}



%%%%% This is the basic vignette experiment %%%%%
%
%\section{Experiment 1: Vignettes}
%In Experiment 1, participants made attributions of emotion to characters in ``near-miss" or ``just-hit" vignettes.
%
%\subsubsection{Materials.} We generated 6 vignettes that involved a dimension of ``distance" which was relevant to the outcome. There were two characters in each vignette. For example, in the ``freeGift" vignette, participants read about Scott and Frank who were both in line for a free brand new vacuum cleaner. Scott was 2nd in line when they ran out of free gifts, while Frank was number 10 when they ran out of free gifts. The possible distances shown were drawn from the following possible values: \{-50, -10, -2, 2, 10, 50\}, where negative values indicates not making the desired outcome. The example described above had distances of -10 and -2. The other vignettes involved: 2) just missing a plane, 3) getting tickets for a concert, 4) making it to a gelato shop before closing time, 5) getting admitted to a course at the local community college and 6) running out of paint when painting a fence.
%
%\subsubsection{Participants.} We recruited 100 participants through Amazon's Mechanical Turk and paid them for completing the experiment. All experiments reported in this paper were conducted according to guidelines approved by the Institutional Review Board at Stanford University.
%
%\subsubsection{Procedures.} Participants viewed each of the six vignettes in a random order. After each vignette, they answered several attention check questions, before rating how each character in the story felt, along six emotions: \{happiness, sadness, contentment, disappointment, relief, and regret\}. After they attributed emotions to both characters, participants made a forced choice rating: which character felt happier? They were also given a neutral option: ``Both feel equally happy".
%
%
%\begin{figure}[htb!]
%\includegraphics[width=\columnwidth]{images/vignettes_forcedHappy.png}
%\caption{ Expt 1 Results. Proportions of forced choice response to ``Who feels happier?". Error bars indicate standard errors. Answers are labeled with the distance of the character's outcome. E.g., in the right-most panel, more people judged the character who just made the outcome (+2) to be happier than the character who made it by a larger margin (+10).}
%\label{Expt1ResultFig}
%\end{figure}
%
%%%%% TODO XXX STATS
%
%\subsubsection{Results.} Post-hoc analysis showed that results were similar across five of the vignettes\footnote{The only anomaly was the fence painting vignette. In the critical comparison in that vignette, X ran out of paint with just 2 inches of fence left, while Y ran out with 10 inches left. Although we predicted that the near-miss character (X) would feel worse, painting a fence involves a persistent product rather than a once-off event: X could return to fence in the future, and he only needs to paint 2 more inches then. In this case, the more distance, the better. We excluded this vignette from the analyses.}: the character that experienced the near-miss (-2) was rated to feel worse than the character that experienced a further-miss (-10, -50) (TODO XXX STATS), and the character that just made the outcome (+2) was rated to feel better than the character that made the outcome by a large margin (+10, +50) (TODO XXX STATS).
%
%The forced choice results are given in Fig. \ref{Expt1ResultFig}. Across the win-lose comparison (left two panels), all participants rated the character who obtained the outcome to be happier. Across lose-lose comparisons, far fewer participants rated the near-miss character as feeling happier (TODO XXX STATS), and across win-win comparisons, far more participants rated the just-hit character as feeling happier (TODO XXX STATS). We note that a large proportion of participants chose the neutral option, and this proportion is higher than than the near-miss/just-hit judgments. 
%
%% 
%\textcolor{red}{\textit{COMMENT: do we interpret the high proportion of "Both" as meaning that it's a weak effect? or that there are individual differences, and majority of Ps might not be sensitive to this?}}
%
%
%%%%%% End Vignette Experiment


\end{document}
