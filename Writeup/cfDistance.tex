% 
% Annual Cognitive Science Conference
% Sample LaTeX Paper -- Proceedings Format
% 

% Original : Ashwin Ram (ashwin@cc.gatech.edu)       04/01/1994
% Modified : Johanna Moore (jmoore@cs.pitt.edu)      03/17/1995
% Modified : David Noelle (noelle@ucsd.edu)          03/15/1996
% Modified : Pat Langley (langley@cs.stanford.edu)   01/26/1997
% Latex2e corrections by Ramin Charles Nakisa        01/28/1997 
% Modified : Tina Eliassi-Rad (eliassi@cs.wisc.edu)  01/31/1998
% Modified : Trisha Yannuzzi (trisha@ircs.upenn.edu) 12/28/1999 (in process)
% Modified : Mary Ellen Foster (M.E.Foster@ed.ac.uk) 12/11/2000
% Modified : Ken Forbus                              01/23/2004
% Modified : Eli M. Silk (esilk@pitt.edu)            05/24/2005
% Modified: Niels Taatgen (taatgen@cmu.edu)  10/24/2006

%% Change ``a4paper'' in the following line to ``letterpaper'' if you are
%% producing a letter-format document.

\documentclass[10pt,letterpaper]{article}

\usepackage{cogsci}
\usepackage{pslatex}
\usepackage{apacite}


\title{ Counterfactual Distance (?) }
 
\author{{\large \bf Desmond C. Ong (dco@stanford.edu)} \\
{\large \bf Jamil Zaki (jzaki@stanford.edu)} \\
{\large \bf Noah D. Goodman (ngoodman@stanford.edu)} \\
  Department of Psychology, Stanford University, Stanford CA, USA 
}

%array of cards
%choose card
%winning card is either proximally closer or numerically closer

%proximally closer vs numerically closer vs different suit (diamond & heart vs. spade & club)
% 3 forced choice (A, B, or equal)

% die + card task

\begin{document}

\maketitle

\begin{abstract}
abstract text

\textbf{Keywords:} 
Near Miss; Counterfactual Distance; Lay Theories; Emotion
\end{abstract}


\begin{quote}
\textit{``Close only counts in horseshoes and hand grenades"} 
-- English Idiom
\end{quote}

	Though we live in only one of many possible realizations of the world, our mental lives---and consequently, our emotional lives---are constantly spent exploring other possible worlds \cite{Bryne2002, Gleicher1990, Johnson1986, Roese1997}. This is done by considering counterfactuals, or what if�s: What if I had woken up earlier (then I would not have missed my train)? What if I had not made that snarky remark to my significant other?

Some worlds are closer than others...


	Counterfactual comparisons, and ``near-miss" counterfactual comparisons in particular, are so mentally engaging is because many of these possible worlds almost happened. Consider \citeA{Kahneman1982}'s classic example of missing a plane by 5 minutes, as opposed to 30 minutes: people reliably judge the person who narrowly missed his plane to feel much worse than the one who missed it by a wider margin. One proposed reason is that it is much easier to generate possible counterfactual antecedents that would have resulted in the counterfactual consequent of catching the plane (``If only I woke up 5 minutes earlier; if only I had packed my bag the night before"). That is, if the counterfactual world is ``closer" to the current world, then perhaps the counterfactual world would only require a smaller change in the causal chain that led up to the current world in order to be realized.

	Previous research has identified the impact of ``closeness" on counterfactual thought \cite{Kahneman1982, Kahneman1990, Roese1997, Teigen1996}. ``Closeness" not only increases the activation of counterfactual thought (by increasing the salience of the counterfactual world)\cite{Roese1997}, but also amplifies the affective consequence of the counterfactual comparison \cite{Johnson1986, Kahneman1982}. Yet, there has been no formal model of how these various types of distances factor into the counterfactual reasoning process.

In particular, there are 2 main questions that a model should answer:
How are different types of distances related?

Missing a plane by 5 minutes versus 30 \cite{Kahneman1982}

Missing a goal in soccer by 5 inches versus 30

Entering a store just ahead of a person who won a prize for being the one-millionth customer \cite{Roese1997}

A �near miss� on a slot machine (e.g., Clark, Lawrence, Astley-Jones, \& Gray, 2009)

In a game where one has to roll a 6 on a die to win, rolling a 5 versus rolling a 2 \cite{Kahneman1990}

How do these distances factor (quantitatively) into our affective judgments?

Here, I propose a model of a lay theory of counterfactual distance, a distance measure to the counterfactual world, and in particular, how counterfactual distance might factor into the affective judgments that human observers make about possible counterfactual worlds. These distances---temporal distance; physical distance; semantic distance in some mental representation space; etc---measure the separation between the current world and the counterfactual world being considered. The model proposes how observers consider these other possible worlds and details the affective impact that counterfactual distance has. 



\section{Acknowledgments}

This work was supported by a James S. McDonnell Foundation Scholar Award to NDG.


\bibliographystyle{apacite}

\setlength{\bibleftmargin}{.125in}
\setlength{\bibindent}{-\bibleftmargin}

\bibliography{cfDistance}


\end{document}
